\documentclass[polish,a4paper,11pt]{mwart}

\usepackage[polish, english]{babel}
\usepackage[utf8]{inputenc}
\usepackage{polski}
\usepackage[T1]{fontenc}
\usepackage{lmodern}  % zestaw fontów
\usepackage{indentfirst}
\frenchspacing

\usepackage{enumerate}
\usepackage{graphicx}
\usepackage{float}
\usepackage{makecell}
\usepackage{siunitx}
\sisetup{output-decimal-marker = {,}}
\usepackage{icomma}
\let\lll\undefined
\usepackage{amsmath, amssymb, amsfonts}
\usepackage{mathtools}
\usepackage{import}		% wklejanie pdf_tex
\usepackage{xcolor}		% kolory
\usepackage{microtype}

\usepackage{csquotes}
\DeclareQuoteAlias{croatian}{polish}

\usepackage{placeins}	% poprawia float

\let\Oldsection\section
\renewcommand{\section}{\FloatBarrier\Oldsection}

\let\Oldsubsection\subsection
\renewcommand{\subsection}{\FloatBarrier\Oldsubsection}

\let\Oldsubsubsection\subsubsection
\renewcommand{\subsubsection}{\FloatBarrier\Oldsubsubsection}

\AtBeginDocument{
  \renewcommand{\tablename}{Tab.}
  \renewcommand{\figurename}{Rys.}
}

\DeclareSIUnit\decibelV{dBV}

\begin{document}


	\begin{table}[h] %Tabelka
	\centering
		\begin{tabular}{ | c |  >{\centering\arraybackslash}m{5.5cm} | c | }
			\hline
			\makecell{ \textbf{Wydział:} \\ IMiR \\ \textbf{Rok:}~5 \\ Semestr: 2 } &
			\textbf{\large{Praktyczne Metody Redukcji Drgań}} &
			\makecell{Data \\ wykonania \\ ćwiczenia: \\ 22.11.2018} \\ \hline
      \makecell{\emph{Wykonujący ćw.:} \\ Marcel Piszak \\ Szymon Mikulicz \\ Zuzanna Kusal \\ Anna Warowny} &
			\large{Aktywna redukcja drgań belki utwierdzonej jednostronnie} &
			\makecell{Nr ćwiczenia: \\ 1} \\ \hline
		\end{tabular}
	\end{table}

  \section{Cel ćwiczenia}

  Celem ćwiczenia była redukcja drgań belki jednostronnie utwierdzonej metodą
  aktywną z użyciem elementów piezoelektrycznych.

  \section{Identyfikacja drgań belki}

  \section{Opis przebiegu ćwiczenia}

  Belkę utwierdzono jednostronnie. Umieszczono na niej wzbudnik, dwa czujniki
  oraz dwa aktuatory. Oprogramowanie sterujące napisano w oprogramowaniu
  LabVIEW\textregistered. Do akwizycji i generacji sygnałów użyto karty
  National Instruments\textregistered. Sygnały na wzbudnik oraz na aktuatory
  podano na wzmacniacze. 

  Dla każdej częstotliości drgań własnych belki przeprowadzano redukcję osobno.
  Ponieważ użyto identycznych czujników oraz przedmiotem zainteresowania były
  względne wartości przyspieszenia drgań, pominięto wyznaczenie wartości
  absolutnych i jako jednostek użyto poziomu napięcia (\si{\decibelV})
  generowanego przez czujniki w odniesieniu do napięcia \SI{1}{\volt} (wartości
  skutecznej). 

  Pierwszy pomiar przeprowadzono dla wyłączonych aktuatorów w celu uzyskania
  inforamcji o amplitudzie drgań bez redukcji. Następnie przeprowadzono
  redukcję za pomocą aktuatorów $A1$, $A2$ oraz $A1+A2$. Każda redukcja była
  przeprowadzana względem czujników $S1$, $S2$ oraz ich sumy $S1+S2$.
  Procedura polegała na takim doborze amplitudy oraz
  przesunięcia fazowego sygnału podawanego na aktuator, aby uzyskać minimum
  dgań w zakresie redukcji.

  \section{Wyniki redukcji}
  
\end{document}
