\documentclass[polish,a4paper,11pt]{mwart}

\usepackage[polish, english]{babel}
\usepackage[utf8]{inputenc}
\usepackage{polski}
\usepackage[T1]{fontenc}
\usepackage{lmodern}  % zestaw fontów
\usepackage{indentfirst}
\frenchspacing

\usepackage{enumerate}
\usepackage{graphicx}
\usepackage{float}
\usepackage{makecell}
\usepackage{siunitx}
\sisetup{output-decimal-marker = {,}}
\usepackage{icomma}
\let\lll\undefined
\usepackage{amsmath, amssymb, amsfonts}
\usepackage{multirow}
\usepackage{mathtools}
\usepackage{import}		% wklejanie pdf_tex
\usepackage{xcolor}		% kolory
\usepackage{microtype}

\usepackage{csquotes}
\DeclareQuoteAlias{croatian}{polish}

\usepackage{placeins}	% poprawia float

\let\Oldsection\section
\renewcommand{\section}{\FloatBarrier\Oldsection}

\let\Oldsubsection\subsection
\renewcommand{\subsection}{\FloatBarrier\Oldsubsection}

\let\Oldsubsubsection\subsubsection
\renewcommand{\subsubsection}{\FloatBarrier\Oldsubsubsection}

\AtBeginDocument{
  \renewcommand{\tablename}{Tab.}
  \renewcommand{\figurename}{Rys.}
}

\DeclareSIUnit\decibelV{dBV}

\begin{document}


	\begin{table}[h] %Tabelka
	\centering
		\begin{tabular}{ | c |  >{\centering\arraybackslash}m{5.5cm} | c | }
			\hline
			\makecell{ \textbf{Wydział:} \\ IMiR \\ \textbf{Rok:}~5 \\ Semestr: 2 } &
			\textbf{\large{Praktyczne Metody Redukcji Drgań}} &
			\makecell{Data \\ wykonania \\ ćwiczenia: \\ 22.11.2018} \\ \hline
      \makecell{\emph{Wykonujący ćw.:} \\ Marcel Piszak \\ Szymon Mikulicz \\ Zuzanna Kusal \\ Anna Warowny} &
			\large{Aktywna redukcja drgań belki utwierdzonej jednostronnie} &
			\makecell{Nr ćwiczenia: \\ 1} \\ \hline
		\end{tabular}
	\end{table}

  \section{Cel ćwiczenia}

  Celem ćwiczenia była redukcja drgań belki jednostronnie utwierdzonej metodą
  aktywną z użyciem elementów piezoelektrycznych.

  \section{Identyfikacja drgań belki}

  Znając materiał, wymiary oraz sposób utwierdzenia belki w sposób teoretyczny obliczono
  cztery pierwsze częstotliwości drgań własnych (Tab. \ref{tab:fwlas}).

  \begin{table}[!tbh]
    \centering
    \caption{Częstotliwości czterech pierwszych modów drgań belki}
    \begin{tabular}{|c|c|c|c|c|}
      \hline
      Numer modu & 1 & 2 & 3 & 4 \\
      \hline
      $f_{wlas} [\si{\hertz}]$ & 110 & 307 & 597 & 991 \\
      \hline
    \end{tabular}
    \label{tab:fwlas}
  \end{table}

  \section{Opis przebiegu ćwiczenia}

  Belkę utwierdzono jednostronnie. Umieszczono na niej wzbudnik, dwa czujniki
  oraz dwa aktuatory. Oprogramowanie sterujące napisano w oprogramowaniu
  LabVIEW\textregistered. Do akwizycji i generacji sygnałów użyto karty
  National Instruments\textregistered. Sygnały na wzbudnik oraz na aktuatory
  podano na wzmacniacze. Schemat układu przedstawiono na rysunku \ref{fig:beam}.

  \begin{figure}[!tbh]
    \centering
    \input{beam.pdf_tex}
    \caption{Schemat układu do redukcji drgań belki}
    \label{fig:beam}
  \end{figure}

  Dla każdej częstotliości drgań własnych belki przeprowadzano redukcję osobno.
  Ponieważ użyto identycznych czujników oraz przedmiotem zainteresowania były
  względne wartości przyspieszenia drgań, pominięto wyznaczenie wartości
  absolutnych i jako jednostek użyto poziomu napięcia (\si{\decibelV})
  generowanego przez czujniki w odniesieniu do napięcia \SI{1}{\volt} (wartości
  skutecznej). 

  Pierwszy pomiar przeprowadzono dla wyłączonych aktuatorów w celu uzyskania
  inforamcji o amplitudzie drgań bez redukcji. Następnie przeprowadzono
  redukcję za pomocą aktuatorów $A1$, $A2$ oraz $A1+A2$. Każda redukcja była
  przeprowadzana względem czujników $S1$, $S2$ oraz ich sumy $S1+S2$.
  Procedura polegała na takim doborze amplitudy oraz
  przesunięcia fazowego sygnału podawanego na aktuator, aby uzyskać minimum
  dgań w zakresie redukcji. Amplituda napięcia podawanego na wzbudnik była
  ustawionana na \SI{500}{\milli\volt}.

  \section{Wyniki redukcji}
  
  \begin{table}[!tbh]
    \centering
    \caption{Wyniki redukcji drgań dla modu nr 1 (\SI{110}{\hertz})}
    \label{tab:red1}
    \begin{tabular}{|c|c|c|c|c|c|c|}
      \cline{3-7}
      \multicolumn{2}{c|}{}&$A1$&$A2$&$S1$&$S2$&$S1+S2$\\\hline
      \multirow{2}{*}{\textit{a priori}} & $A$ & - & - & -28,7 & -40,2 & -28,8\\\cline{2-7}
					 & $\Phi$ & - & - & \multicolumn{3}{c}{}\\\hline
      \multirow{6}{*}{$\min\{S1\}$}   &   $A$ & 0,26 & - & -28,7 & -66 & -28,8\\\cline{2-7}
				      &$\Phi$ & 345 & - & \multicolumn{3}{c}{}\\\cline{2-7}
				      &   $A$ & - & 4,5 & -34,8 & -18,9 & -18,9\\\cline{2-7}
				      &$\Phi$ & - & 185 & \multicolumn{3}{c}{}\\\cline{2-7}
				      &   $A$ & 0,26 & 0,15 & -80 & -47,2 & -47,2\\\cline{2-7}
				      &$\Phi$ & 345 & 203 & \multicolumn{3}{c}{}\\\hline
      \multirow{6}{*}{$\min\{S2\}$}   &   $A$ & 0,27 & - & -55,2 & -83 & -55,5\\\cline{2-7}
				      &$\Phi$ & 348 & - & \multicolumn{3}{c}{}\\\cline{2-7}
				      &   $A$ & - & 0,3 & -29,3 & -61,5 & -29,3\\\cline{2-7}
				      &$\Phi$ & - & 229 & \multicolumn{3}{c}{}\\\cline{2-7}
				      &   $A$ & 0,27 & 0,15 & -35,4 & -76,2 & -35,4\\\cline{2-7}
				      &$\Phi$ & 14 & 149 & \multicolumn{3}{c}{}\\\hline
      \multirow{6}{*}{$\min\{S1+S2\}$}&   $A$ & 0,26 & - & -62,5 & -65,2 & -63,1\\\cline{2-7}
				      &$\Phi$ & 346 & - & \multicolumn{3}{c}{}\\\cline{2-7}
				      &   $A$ & - & 1,2 & -30,8 & -31,7 & -30,8\\\cline{2-7}
				      &$\Phi$ & - & 204 & \multicolumn{3}{c}{}\\\cline{2-7}
				      &   $A$ & 0,26 & 0 & -57,5 & -66 & -57,7\\\cline{2-7}
				      &$\Phi$ & 346 & 0 & \multicolumn{3}{c}{}\\\cline{1-4}
    \end{tabular}
  \end{table}

  \begin{table}[!tbh]
    \centering
    \caption{Wyniki redukcji drgań dla modu nr 2 (\SI{307}{\hertz})}
    \label{tab:red2}
    \begin{tabular}{|c|c|c|c|c|c|c|}
      \cline{3-7}
      \multicolumn{2}{c|}{}&$A1$&$A2$&$S1$&$S2$&$S1+S2$\\\hline
      \multirow{2}{*}{\textit{a priori}} & $A$ & - & - & -25,1 & -37,3 & -25,1\\\cline{2-7}
					 & $\Phi$ & - & - & \multicolumn{3}{c}{}\\\hline
      \multirow{6}{*}{$\min\{S1\}$}   &   $A$ & 4,5 & - & -39,6 & -14,4 & -14,4\\\cline{2-7}
				      &$\Phi$ & 0 & - & \multicolumn{3}{c}{}\\\cline{2-7}
				      &   $A$ & - & 0,53 & -46,9 & -35,9 & -35,9\\\cline{2-7}
				      &$\Phi$ & - & 0 & \multicolumn{3}{c}{}\\\cline{2-7}
				      &   $A$ & 0,4 & 0,5 & -65,4 & -30,9 & -30,9\\\cline{2-7}
				      &$\Phi$ & 52 & 0 & \multicolumn{3}{c}{}\\\hline
      \multirow{6}{*}{$\min\{S2\}$}   &   $A$ & 0,33 & - & -24,6 & -75,0 & -24,6\\\cline{2-7}
				      &$\Phi$ & 197 & - & \multicolumn{3}{c}{}\\\cline{2-7}
				      &   $A$ & - & 1,3 & -15,8 & -67,3 & -15,8\\\cline{2-7}
				      &$\Phi$ & - & 254 & \multicolumn{3}{c}{}\\\cline{2-7}
				      &   $A$ & 0,32 & 1,3 & -14,5 & -74,8 & -14,5\\\cline{2-7}
				      &$\Phi$ & 261 & 197 & \multicolumn{3}{c}{}\\\hline
      \multirow{6}{*}{$\min\{S1+S2\}$}&   $A$ & 0,77 & - & -26,7 & -27,2 & -26,6\\\cline{2-7}
				      &$\Phi$ & 254 & - & \multicolumn{3}{c}{}\\\cline{2-7}
				      &   $A$ & - & 0,4 & -37,3 & -36,3 & -36,3\\\cline{2-7}
				      &$\Phi$ & - & 0 & \multicolumn{3}{c}{}\\\cline{2-7}
				      &   $A$ & 0,3 & 0,6 & -47,3 & -45,7 & -45,7\\\cline{2-7}
				      &$\Phi$ & 186 & 4 & \multicolumn{3}{c}{}\\\cline{1-4}
    \end{tabular}
  \end{table}

  \begin{table}[!tbh]
    \centering
    \caption{Wyniki redukcji drgań dla modu nr 3 (\SI{597}{\hertz})}
    \label{tab:red3}
    \begin{tabular}{|c|c|c|c|c|c|c|}
      \cline{3-7}
      \multicolumn{2}{c|}{}&$A1$&$A2$&$S1$&$S2$&$S1+S2$\\\hline
      \multirow{2}{*}{\textit{a priori}} & $A$ & - & - & -14 & -23,8 & -14\\\cline{2-7}
					 & $\Phi$ & - & - & \multicolumn{3}{c}{}\\\hline
      \multirow{6}{*}{$\min\{S1\}$}   &   $A$ & 3,16 & - & -78 & -16,3 & -16,3\\\cline{2-7}
				      &$\Phi$ & 150 & - & \multicolumn{3}{c}{}\\\cline{2-7}
				      &   $A$ & - & 1,6 & -46,8 & -22,6 & -22,6\\\cline{2-7}
				      &$\Phi$ & - & 256 & \multicolumn{3}{c}{}\\\cline{2-7}
				      &   $A$ & 0,1 & 1,6 & -61,1 & -23,3 & -23,3\\\cline{2-7}
				      &$\Phi$ & 87 & 257 & \multicolumn{3}{c}{}\\\hline
      \multirow{6}{*}{$\min\{S2\}$}   &   $A$ & 1,47 & - & -13,1 & -85,2 & -13,1\\\cline{2-7}
				      &$\Phi$ & 58 & - & \multicolumn{3}{c}{}\\\cline{2-7}
				      &   $A$ & - & 2,3 & -8,4 & -31,9 & -8,4\\\cline{2-7}
				      &$\Phi$ & - & 0 & \multicolumn{3}{c}{}\\\cline{2-7}
				      &   $A$ & 1,5 & 0,1 & -13,4 & -68,6 & -13,4\\\cline{2-7}
				      &$\Phi$ & 59 & 198 & \multicolumn{3}{c}{}\\\hline
      \multirow{6}{*}{$\min\{S1+S2\}$}&   $A$ & 1,7 & - & -20,9 & -20,8 & -20,5\\\cline{2-7}
				      &$\Phi$ & 143 & - & \multicolumn{3}{c}{}\\\cline{2-7}
				      &   $A$ & - & 1,4 & -20,8 & -24,2 & -24,1\\\cline{2-7}
				      &$\Phi$ & - & 271 & \multicolumn{3}{c}{}\\\cline{2-7}
				      &   $A$ & 1,1 & 1,4 & -20,5 & -39,9 & -39,8\\\cline{2-7}
				      &$\Phi$ & 100 & 276 & \multicolumn{3}{c}{}\\\cline{1-4}
    \end{tabular}
  \end{table}

  \begin{table}[!tbh]
    \centering
    \caption{Wyniki redukcji drgań dla modu nr 4 (\SI{991}{\hertz})}
    \label{tab:red4}
    \begin{tabular}{|c|c|c|c|c|c|c|}
      \cline{3-7}
      \multicolumn{2}{c|}{}&$A1$&$A2$&$S1$&$S2$&$S1+S2$\\\hline
      \multirow{2}{*}{\textit{a priori}} & $A$ & - & - & -12,8 & -22 & -12,8\\\cline{2-7}
					 & $\Phi$ & - & - & \multicolumn{3}{c}{}\\\hline
      \multirow{6}{*}{$\min\{S1\}$}   &   $A$ & 3,3 & - & -52,2 & -15,9 & -15,9\\\cline{2-7}
				      &$\Phi$ & 324 & - & \multicolumn{3}{c}{}\\\cline{2-7}
				      &   $A$ & - & 2,1 & -50,7 & -18,9 & -18,9\\\cline{2-7}
				      &$\Phi$ & - & 236 & \multicolumn{3}{c}{}\\\cline{2-7}
				      &   $A$ & 3,2 & 0,09 & -82,4 & -16,1 & -16,1\\\cline{2-7}
				      &$\Phi$ & 325 & 210 & \multicolumn{3}{c}{}\\\hline
      \multirow{6}{*}{$\min\{S2\}$}   &   $A$ & 2,6 & - & -10 & -34,5 & -10\\\cline{2-7}
				      &$\Phi$ & 218 & - & \multicolumn{3}{c}{}\\\cline{2-7}
				      &   $A$ & - & 1,9 & -10,5 & -66,3 & -10,5\\\cline{2-7}
				      &$\Phi$ & - & 323 & \multicolumn{3}{c}{}\\\cline{2-7}
				      &   $A$ & 0,1 & 2 & -10,2 & -83 & -10,2\\\cline{2-7}
				      &$\Phi$ & 46 & 323 & \multicolumn{3}{c}{}\\\hline
      \multirow{6}{*}{$\min\{S1+S2\}$}&   $A$ & 1,7 & - & -21,4 & -21,8 & -21,2\\\cline{2-7}
				      &$\Phi$ & 319 & - & \multicolumn{3}{c}{}\\\cline{2-7}
				      &   $A$ & - & 1,8 & -21,4 & -21,8 & -21,2\\\cline{2-7}
				      &$\Phi$ & - & 258 & \multicolumn{3}{c}{}\\\cline{2-7}
				      &   $A$ & 2 & 2 & -36,1 & -30,4 & -30,4\\\cline{2-7}
				      &$\Phi$ & 254 & 263 & \multicolumn{3}{c}{}\\\cline{1-4}
    \end{tabular}
  \end{table}

\end{document}
